%%%%%%%%%%%%%%%%%%%%%%%%%%%%%%
%  Work Package Description  %
%%%%%%%%%%%%%%%%%%%%%%%%%%%%%%

\begin{workpackage}{Model 2 n Commands}
  \label{wp:m2cN} %change and use appropriate description

  %%%%%%%%%%%%%%%%%% TOP TABLE %%%%%%%%%%%%%%%%%%%%%%%%%%%%%
  % Data for the top table
  \wpstart{12} %Starting Week
  \wpend{18} %End Week
  \wptype{Activity type} %RTD, DEM, MGT, or OTHER

  % Person Weeks per participant (required, max 7, * for leader)  
  % syntax: \personweeks{Participant number}{value}    (not wp leader)
  %     or  \personweeks{Participant short name}{value} (not wp leader)
  %         \personweeks*{Participant number}{value}    (wp leader)
  % for example:
  \personweeks*{Georgio}{6}
  % etc.

  \makewptable % Work package summary table
    
  % Work Package Objectives
  \begin{wpobjectives}
    This work package has the following objectives:
    \begin{enumerate}
    \item Replicate all steps in WP2 and WP3 so that we get a Model 2 for a different individual trained on n the same n Commands as Model 1;
    \item Play a game of 2P Super Mario Bros.
    \end{enumerate}
  \end{wpobjectives}
  
  % Work Package Description
  \begin{wpdescription}
    % Divide work package into multiple tasks.
    % Use \wptask command
    % syntax: \wptask{leader}{contributors}{start-m}{end-m}{title}{description}   
 
    \wptask{Georgio}{Georgio}{12}{18}{Task1}{
      \label{task:wp4task1} 
      A new Model 2 will be created, repeting the steps from previous WPs, in such a way that it is trained using data gathered from a different individual for the same n-Commands.}
      \wptask{Georgio}{Georgio}{12}{13}{Task2}{
        \label{task:wp4task2}
       The ability to switch between models will be added to the virtualHID code, for testing purposes.
      }   
      \wptask{Georgio}{Georgio}{18}{18}{Task3}{
      \label{task:wp4task3}
     Models 1 and 2 will be published.
    } \wptask{Georgio}{Georgio}{18}{18}{Task4}{
      \label{task:wp4task4}
     A game of 2P Super Mario Bros will be played.
    }    

    
  \end{wpdescription}
  
  % Work Package Deliverable
  \begin{wpdeliverables}
    % Data for the deliverables and milestones  tables
    % syntax: \deliverable[delivery date]{nature}{dissemination
    % level}{description} 
    %
    % nature: R = Report, P = Prototype, D = Demonstrator, O = Other
    % dissemination level: PU = Public, PP = Restricted to other
    % programme participants (including the Commission Services), RE =
    % Restricted to a group specified by the consortium (including the
    % Commission Services), CO = Confidential, only for members of the
    % consortium (including the Commission Services).
    % 
    % \wpdeliverable[date]{R}{PU}{A report on \ldots}


    \wpdeliverable[13]{Georgio}{P}{PU}{Merging model switching code to master.}\label{dev:wp4merge1}
    \wpdeliverable[13]{Georgio}{R}{PU}{Report on the ability to use 2 models simultaneously as 2 virtualHID devices.}\label{dev:wp4r1}
    \wpdeliverable[15]{Georgio}{R}{PU}{Report on Model 2 Command 8.}\label{dev:wp4r2}
    \wpdeliverable[18]{Georgio}{R}{PU}{Report on training Model 2 for n-Commands.}\label{dev:wp4r3}
    \wpdeliverable[13]{Georgio}{P}{PU}{Merging code of Models 1 and 2 to master.}\label{dev:wp4merge2}
    \wpdeliverable[18]{Georgio}{D}{PU}{A demonstration of the ability to play a game of 2P Super Mario Bros using cogniLink.}\label{dev:wp4vid}
    \wpdeliverable[18]{Georgio}{R}{PU}{Main Report update with full progress accomplished after the end of WP4.}\label{dev:MR0.4}
    
  \end{wpdeliverables}
\end{workpackage}
%%% Local Variables:
%%% mode: latex
%%% TeX-master: "proposal-main"
%%% End:
