\begin{abstract}
Although great strides have been achieved in making 
computers more accessible, it’s indisputable that 
there remains huge prospects for improvement. Given that 
technology is designed for the masses, it offers every 
individual a platform to do what is needed; this 
includes, but isn’t limited to, support for 
individuals with motor, dexterity, and/or speech 
impairments.  In this proposal, we will discuss 
cogniLink, a tool that assists developers in making 
computers more accessible for persons with afflictions. 
cogniLink is a brain-computer interface that allows the 
user to trigger the execution of a command simply by 
thinking of the trigger. A training data set is to be 
collected from n-users in order to train n-models using 
an ElectroEncephaloGram (EEG). Each model is programmed 
to recognize one or more trigger thoughts. The same model 
interacts with a stack of software which allows it to map 
positive outputs from the model and transform it into an 
actionable command. For the purpose of demonstration, the 
model will be trained to recognize commands from one user 
which will be mapped to a virtualHID in such a way that 
allows the user to play Super Mario Bros. After an 
extensive process of training n-models, a universal model 
(UM) will be trained using data from the aforementioned 
n-models in order to have a simpler training process for 
new users. cogniLink will allow disabled people to 
execute commands in a very seamless and orderly fashion, 
thus making computers more accessible to persons with 
digital input impairments. Two of cogniLink’s long 
term goals are to allow an amputee to be able to 
effortlessly be able to control a wheelchair in real 
time, and for someone suffering from Locked-in Syndrome 
to be able to interact with the world around them with 
ease.
\end{abstract}
