%%%%%%%%%%%%%%%%%%%%%%%%%%%%%%
%  Work Package Description  %
%%%%%%%%%%%%%%%%%%%%%%%%%%%%%%

\begin{workpackage}{Model 1 Command A}
  \label{wp:m1cA} %change and use appropriate description

  %%%%%%%%%%%%%%%%%% TOP TABLE %%%%%%%%%%%%%%%%%%%%%%%%%%%%%
  % Data for the top table
  \wpstart{2} %Starting Week
  \wpend{6} %End Week
  \wptype{Activity type} %RTD, DEM, MGT, or OTHER

  % Person Weeks per participant (required, max 7, * for leader)  
  % syntax: \personweeks{Participant number}{value}    (not wp leader)
  %     or  \personweeks{Participant short name}{value} (not wp leader)
  %         \personweeks*{Participant number}{value}    (wp leader)
  % for example:
  \personweeks*{Georgio}{4}  % etc.

  \makewptable % Work package summary table
    
  % Work Package Objectives
  \begin{wpobjectives}
    \begin{enumerate}
    \item To collect training, validation, and test datasets for 
Model 1 Command A; 
    \item Training Model 1 using aforementioned data;
    \item Testing/Patching Model 1.
    \end{enumerate}
  \end{wpobjectives}
  
  % Work Package Description
  \begin{wpdescription}
    % Divide work package into multiple tasks.
    % Use \wptask command
    % syntax: \wptask{leader}{contributors}{start-m}{end-m}{title}{description}   

    \wptask{Georgio}{Georgio}{2}{3}{Task1}{
      \label{task:wp2task1}
      Ways to efficiently collect data with high accuracy will be looked into; if found, validated datasets will be used as a point of reference.
    }
    \wptask{Georgio}{Georgio}{3}{4}{Task2}{
      \label{task:wp2task2}
      The training dataset will be collected.
    }    
    \wptask{Georgio}{Georgio}{4}{5}{Task 3}{
      \label{task:wp2task3}
      Model 1 will be trained using the aforementioned dataset.
    }
    \wptask{Georgio}{Georgio}{4}{5}{Task 4}{
      \label{task:wp2task4}
      Test and Validation datasets will be collected.
      }
      \wptask{Georgio}{Georgio}{4}{5}{Task 5}{
      \label{task:wp2task4}
      Test and Validation datasets will be collected.
      }
      \wptask{Georgio}{Georgio}{4}{5}{Task 6}{
      \label{task:wp2task6}
      All collected datasets will be uploaded to an AWS S3 Bucket.
      }
      \wptask{Georgio}{Georgio}{4}{5}{Task 7}{
        \label{task:wp2task7}
        Accuracy of trained model will be studied.
        }
      \wptask{Georgio}{Georgio}{5}{6}{Task 8}{
        \label{task:wp2task8}
        Patches and optimizations will be pushed in attempt to improve model accuracy, if possible.
        }
   
  \end{wpdescription}
  
  % Work Package Deliverable
  \begin{wpdeliverables}
    % Data for the deliverables and milestones  tables
    % syntax: \deliverable[delivery date]{nature}{dissemination
    % level}{description} 
    %
    % nature: R = Report, P = Prototype, D = Demonstrator, O = Other
    % dissemination level: PU = Public, PP = Restricted to other
    % programme participants (including the Commission Services), RE =
    % Restricted to a group specified by the consortium (including the
    % Commission Services), CO = Confidential, only for members of the
    % consortium (including the Commission Services).
    % 
    % \wpdeliverable[date]{R}{PU}{A report on \ldots}

    \wpdeliverable[4]{Georgio}{R}{PU}{Report 1 about the data collection process and initiation of the first round of training.}\label{dev:wp2r1}
    \wpdeliverable[5]{Georgio}{P}{PU}{First iteration of the model.}\label{dev:wp2m0.1}
    \wpdeliverable[5]{Georgio}{R}{PU}{Report 2 about model accuracy after inputing initial test and validation datasets.}\label{dev:wp2r2}
    \wpdeliverable[6]{Georgio}{R}{PU}{Report 3 will include a comparative view of accuracy for each patch/iteration of the model.}\label{dev:wp2r3}
    \wpdeliverable[6]{Georgio}{P}{PU}{Second iteration of the model.}\label{dev:wp2m0.2}
    \wpdeliverable[6]{Georgio}{R}{PU}{Main Report update with full progress accomplished after the end of WP2.}\label{dev:MR0.2}
    \wpdeliverable[6]{Georgio}{D}{PU}{Video demonstration of thought recognition process.}\label{dev:vid1}
  \end{wpdeliverables}

\end{workpackage}


%%% Local Variables:
%%% mode: latex
%%% TeX-master: "proposal-main"
%%% End:
