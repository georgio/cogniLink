\chapter{Implementation}
\label{cha:implementation}

\section{Tool Chain}
\label{sec:tc}
\begin{figure}[h!]
  \includegraphics[width=\linewidth]{diagram.jpg}
  \caption{Tool Chain Diagram}
  \label{fig:tcd}
\end{figure}

Text about tool chain here
Figure \ref{fig:tcd}.

\section{Project work plan}
\label{sec:work-plan}

% Gantt chart in latex (requires pfggantt.sty file)
%\include{gantt}

\subsection{Work package description}
\label{sec:wps}

%Include work-packages as separate files
%%%%%%%%%%%%%%%%%%%%%%%%%%%%%%
%  Work Package Description  %
%%%%%%%%%%%%%%%%%%%%%%%%%%%%%%

\begin{workpackage}{Virtual HID, Data Collection, and ML Code}
  \label{wp:data} %change and use appropriate description

  %%%%%%%%%%%%%%%%%% TOP TABLE %%%%%%%%%%%%%%%%%%%%%%%%%%%%%
  % Data for the top table
  \wpstart{1} %Starting Week
  \wpend{2} %End Week
  \wptype{Activity type} %RTD, DEM, MGT, or OTHER

  % Person Weeks per participant (required, max 7, * for leader)  
  % syntax: \personweeks{Participant number}{value}    (not wp leader)
  %     or  \personweeks{Participant short name}{value} (not wp leader)
  %         \personweeks*{Participant number}{value}    (wp leader)
  % for example:
  \personweeks*{georgio}{2}
  % etc.

  \makewptable % Work package summary table
    
  % Work Package Objectives
  \begin{wpobjectives}
    This work package has the following objectives:
    \begin{enumerate}
    \item To develop a Virtual Human Interface Device;
    \item To develop an  API that gathers raw data from the Cyton board and feeds it 
to a CSV file;
    \item To write code needed to efficiently store and manage datasets;
    \item To write code needed to start training Model 1 on Command A.
    \end{enumerate}
  \end{wpobjectives}
  
  % Work Package Description
  \begin{wpdescription}
    % Divide work package into multiple tasks.
    % Use \wptask command
    % syntax: \wptask{leader}{contributors}{start-m}{end-m}{title}{description}   
 
    \wptask{georgio}{georgio}{1}{1}{Task1}{
      \label{task:wp1task1}
      The virtualHID will be created using macOS' IOKit Library.
    }
    \wptask{georgio}{georgio}{1}{1}{Task2}{
      \label{task:wp1task2}
      The Cyton board will be programmed using arsh.
    }    
    \wptask{georgio}{georgio}{1}{1}{Task3}{
      \label{task:wp1task3}
      API to gather data from Cyton Board will be built
    }
    \wptask{georgio}{georgio}{1}{2}{Task4}{
      \label{task:wp1task4}
      Code to manage raw EEG data will be done here.
      }
      \wptask{georgio}{georgio}{2}{2}{Task5}{
      \label{task:wp1task5}
      EC2, S3, and SageMaker instances will be configured.
    }
    \wptask{georgio}{georgio}{2}{2}{Task5}{
      \label{task:wp1task5}
      Code to feed raw data to S3 bucket will be done here.
    }
    \wptask{georgio}{georgio}{2}{2}{Task6}{
      \label{task:wp1task6}
      Code to start training the model will be done here.
    }
  \end{wpdescription}
  
  % Work Package Deliverable
  \begin{wpdeliverables}
    % Data for the deliverables and milestones  tables
    % syntax: \deliverable[delivery date]{nature}{dissemination
    % level}{description} 
    %
    % nature: R = Report, P = Prototype, D = Demonstrator, O = Other
    % dissemination level: PU = Public, PP = Restricted to other
    % programme participants (including the Commission Services), RE =
    % Restricted to a group specified by the consortium (including the
    % Commission Services), CO = Confidential, only for members of the
    % consortium (including the Commission Services).
    % 
    % \wpdeliverable[date]{R}{PU}{A report on \ldots}

    \wpdeliverable[1]{georgio}{R}{PU}{WP1 W1 Progress Report.}\label{wp1:reportw1}
    \wpdeliverable[2]{georgio}{D}{PU}{Demonstration of APIs}\label{wp1:demo}
    \wpdeliverable[2]{georgio}{R}{PU}{End of WP1 - Overall Progress Report.}\label{report:wp1}
    \wpdeliverable[2]{georgio}{P}{PU}{WP1 Code+Tools Merged to master.}\label{wp1:prototype}
  \end{wpdeliverables}
\end{workpackage}
%%% Local Variables:
%%% mode: latex
%%% TeX-master: "proposal-main"
%%% End:
             %Use \input for first WP
%%%%%%%%%%%%%%%%%%%%%%%%%%%%%%
%  Work Package Description  %
%%%%%%%%%%%%%%%%%%%%%%%%%%%%%%

\begin{workpackage}{Model 1 Command A}
  \label{wp:m1cA} %change and use appropriate description

  %%%%%%%%%%%%%%%%%% TOP TABLE %%%%%%%%%%%%%%%%%%%%%%%%%%%%%
  % Data for the top table
  \wpstart{2} %Starting Week
  \wpend{6} %End Week
  \wptype{Activity type} %RTD, DEM, MGT, or OTHER

  % Person Weeks per participant (required, max 7, * for leader)  
  % syntax: \personweeks{Participant number}{value}    (not wp leader)
  %     or  \personweeks{Participant short name}{value} (not wp leader)
  %         \personweeks*{Participant number}{value}    (wp leader)
  % for example:
  \personweeks*{georgio}{4}  % etc.

  \makewptable % Work package summary table
    
  % Work Package Objectives
  \begin{wpobjectives}
    \begin{enumerate}
    \item To collect training, validation, and test datasets for 
Model 1 Command A; 
    \item Training Model 1 using aforementioned data;
    \item Testing/Patching Model 1.
    \end{enumerate}
  \end{wpobjectives}
  
  % Work Package Description
  \begin{wpdescription}
    % Divide work package into multiple tasks.
    % Use \wptask command
    % syntax: \wptask{leader}{contributors}{start-m}{end-m}{title}{description}   

    \wptask{georgio}{georgio}{2}{3}{Task1}{
      \label{task:wp2task1}
      Ways to efficiently collect data with high accuracy will be looked into; validated datasets will be used (if found) as a point of reference.
    }
    \wptask{georgio}{georgio}{3}{4}{Task2}{
      \label{task:wp2task2}
      The training dataset will be collected.
    }    
    \wptask{georgio}{georgio}{4}{5}{Task 3}{
      \label{task:wp2task3}
      Model 1 will be trained using the aforementioned dataset.
    }
    \wptask{georgio}{georgio}{4}{5}{Task 4}{
      \label{task:wp2task4}
      Test and Validation datasets will be collected.
      }
      \wptask{georgio}{georgio}{4}{5}{Task 5}{
      \label{task:wp2task4}
      Test and Validation datasets will be collected.
      }
      \wptask{georgio}{georgio}{4}{5}{Task 6}{
      \label{task:wp2task6}
      All collected datasets will be uploaded to an AWS S3 Bucket.
      }
      \wptask{georgio}{georgio}{4}{5}{Task 7}{
        \label{task:wp2task7}
        Accuracy of trained model will be studied.
        }
      \wptask{georgio}{georgio}{5}{6}{Task 8}{
        \label{task:wp2task8}
        Patches and optimizations will be pushed in attempt to improve model accuracy, if possible.
        }
   
  \end{wpdescription}
  
  % Work Package Deliverable
  \begin{wpdeliverables}
    % Data for the deliverables and milestones  tables
    % syntax: \deliverable[delivery date]{nature}{dissemination
    % level}{description} 
    %
    % nature: R = Report, P = Prototype, D = Demonstrator, O = Other
    % dissemination level: PU = Public, PP = Restricted to other
    % programme participants (including the Commission Services), RE =
    % Restricted to a group specified by the consortium (including the
    % Commission Services), CO = Confidential, only for members of the
    % consortium (including the Commission Services).
    % 
    % \wpdeliverable[date]{R}{PU}{A report on \ldots}

    \wpdeliverable[4]{georgio}{R}{PU}{Report 1 about the data collection process and initiation of the first round of training.}\label{dev:wp2r1}
    \wpdeliverable[5]{georgio}{P}{PU}{First iteration of the model.}\label{dev:wp2m0.1}
    \wpdeliverable[5]{georgio}{R}{PU}{Report 2 about model accuracy after inputing initial test and validation datasets.}\label{dev:wp2r2}
    \wpdeliverable[6]{georgio}{R}{PU}{Report 3 will include a comparative view of accuracy for each patch/iteration of the model.}\label{dev:wp2r3}
    \wpdeliverable[6]{georgio}{P}{PU}{Second iteration of the model.}\label{dev:wp2m0.2}
    \wpdeliverable[6]{georgio}{R}{PU}{Main Report with full progress update after the end of WP2.}\label{dev:MR0.1}      \wpdeliverable[6]{georgio}{D}{PU}{Video demonstration of thought recognition process.}\label{dev:vid1}
  \end{wpdeliverables}

\end{workpackage}


%%% Local Variables:
%%% mode: latex
%%% TeX-master: "proposal-main"
%%% End:

%%%%%%%%%%%%%%%%%%%%%%%%%%%%%%
%  Work Package Description  %
%%%%%%%%%%%%%%%%%%%%%%%%%%%%%%

\begin{workpackage}{Model 1 n Commands}
  \label{wp:m1cN} %change and use appropriate description
  %%%%%%%%%%%%%%%%%% TOP TABLE %%%%%%%%%%%%%%%%%%%%%%%%%%%%%
  % Data for the top table
  \wpstart{6} %Starting Week
  \wpend{12} %End Week
  \wptype{Activity type} %RTD, DEM, MGT, or OTHER

  % Person Weeks per participant (required, max 7, * for leader)  
  % syntax: \personweeks{Participant number}{value}    (not wp leader)
  %     or  \personweeks{Participant short name}{value} (not wp leader)
  %         \personweeks*{Participant number}{value}    (wp leader)
  % for example:
  \personweeks*{georgio}{6}
  % etc.

  \makewptable % Work package summary table
    
  % Work Package Objectives
  \begin{wpobjectives}
    This work package has the following objectives:
    \begin{enumerate*}
    \item Link the output from the Model to the virtual HID created in WP1;
    \item Map trigger thoughts to button presses; 
    \item Play a game of 1P Super Mario Bros.
    \end{enumerate*}
  \end{wpobjectives}
  
  % Work Package Description
  \begin{wpdescription}
    % Divide work package into multiple tasks.
    % Use \wptask command
    % syntax: \wptask{leader}{contributors}{start-m}{end-m}{title}{description}   

    \wptask{georgio}{georgio}{6}{12}{Task1}{
      \label{task:wp3HIDfinilazation}
      Code for linking model to virtual HID will be rechecked and finalized.
    }
    \wptask{georgio}{georgio}{6}{12}{Task2}{
      \label{task:wp3training}
      Model 1 will be trained for command B, and other commands simultaneously.
    }    
    \wptask{georgio}{georgio}{12}{12}{Task3}{
      \label{task:wp3demo}
      A game of 1P Super Mario Bros will be played.
    }    
    
    %%~\ref{task:wp3integrate}.
   
  \end{wpdescription}
  
  % Work Package Deliverable
  \begin{wpdeliverables}
    % Data for the deliverables and milestones  tables
    % syntax: \deliverable[delivery date]{nature}{dissemination
    % level}{description} 
    %
    % nature: R = Report, P = Prototype, D = Demonstrator, O = Other
    % dissemination level: PU = Public, PP = Restricted to other
    % programme participants (including the Commission Services), RE =
    % Restricted to a group specified by the consortium (including the
    % Commission Services), CO = Confidential, only for members of the
    % consortium (including the Commission Services).
    % 
    % \wpdeliverable[date]{R}{PU}{A report on \ldots}

    \wpdeliverable[8]{georgio}{D}{PU}{A demonstration of the execution of alternating commands, after successful training of the second command to model 1.}\label{dev:wp3d1}
    \wpdeliverable[11]{georgio}{R}{PU}{Report 1 on findings made while training new commands.}\label{dev:wp3r1}
    \wpdeliverable[12]{georgio}{R}{PU}{Report 2 will comapre the variation of latency between the model and the virtual HID for each code patch.}\label{dev:wp3r2}
    \wpdeliverable[12]{georgio}{P}{PU}{Code for virtual HID and updated Model 1 with multiple command recognition will be pushed to master.}\label{dev:wp3r1}
    \wpdeliverable[12]{georgio}{D}{PU}{A demonstration of the ability to play a game of 1P Super Mario Bros using cogniLink.}\label{dev:wp3prototype}
    \wpdeliverable[12]{georgio}{R}{PU}{Main Report update with full progress accomplished after the end of WP3.}\label{dev:MR0.2}
  \end{wpdeliverables}

\end{workpackage}


%%% Local Variables:
%%% mode: latex
%%% TeX-master: "proposal-main"
%%% End:

%%%%%%%%%%%%%%%%%%%%%%%%%%%%%%
%  Work Package Description  %
%%%%%%%%%%%%%%%%%%%%%%%%%%%%%%

\begin{workpackage}{Model 2 n Commands}
  \label{wp:m2cN} %change and use appropriate description

  %%%%%%%%%%%%%%%%%% TOP TABLE %%%%%%%%%%%%%%%%%%%%%%%%%%%%%
  % Data for the top table
  \wpstart{12} %Starting Week
  \wpend{18} %End Week
  \wptype{Activity type} %RTD, DEM, MGT, or OTHER

  % Person Weeks per participant (required, max 7, * for leader)  
  % syntax: \personweeks{Participant number}{value}    (not wp leader)
  %     or  \personweeks{Participant short name}{value} (not wp leader)
  %         \personweeks*{Participant number}{value}    (wp leader)
  % for example:
  \personweeks*{georgio}{6}
  % etc.

  \makewptable % Work package summary table
    
  % Work Package Objectives
  \begin{wpobjectives}
    This work package has the following objectives:
    \begin{enumerate}
    \item Replicate all steps in WP2 and WP3 so that we get a Model 2 for a different individual trained on n the same n Commands as Model 1;
    \item Play a game of 2P Super Mario Bros.
    \end{enumerate}
  \end{wpobjectives}
  
  % Work Package Description
  \begin{wpdescription}
    % Divide work package into multiple tasks.
    % Use \wptask command
    % syntax: \wptask{leader}{contributors}{start-m}{end-m}{title}{description}   
 
    \wptask{georgio}{georgio}{12}{18}{Task1}{
      \label{task:wp4task1} 
      A new Model 2 will be created, repeting the steps from previous WPs, in such a way that it is trained using data gathered from a different individual for the same n-Commands.}
      \wptask{georgio}{georgio}{12}{13}{Task2}{
        \label{task:wp4task2}
       The ability to switch between models will be added to the virtual HID code, for testing purposes.
      }   
      \wptask{georgio}{georgio}{18}{18}{Task3}{
      \label{task:wp4task3}
     Models 1 and 2 will be published.
    } \wptask{georgio}{georgio}{18}{18}{Task4}{
      \label{task:wp4task4}
     A game of 2P Super Mario Bros will be played.
    }    

    
  \end{wpdescription}
  
  % Work Package Deliverable
  \begin{wpdeliverables}
    % Data for the deliverables and milestones  tables
    % syntax: \deliverable[delivery date]{nature}{dissemination
    % level}{description} 
    %
    % nature: R = Report, P = Prototype, D = Demonstrator, O = Other
    % dissemination level: PU = Public, PP = Restricted to other
    % programme participants (including the Commission Services), RE =
    % Restricted to a group specified by the consortium (including the
    % Commission Services), CO = Confidential, only for members of the
    % consortium (including the Commission Services).
    % 
    % \wpdeliverable[date]{R}{PU}{A report on \ldots}


    \wpdeliverable[13]{georgio}{P}{PU}{Merging model switching code to master.}\label{dev:wp4merge1}
    \wpdeliverable[13]{georgio}{R}{PU}{Report on the ability to use 2 models simultaneously as 2 virtual HID devices.}\label{dev:wp4r1}
    \wpdeliverable[15]{georgio}{R}{PU}{Report on Model 2 Command 8.}\label{dev:wp4r2}
    \wpdeliverable[18]{georgio}{R}{PU}{Report on training Model 2 for n-Commands.}\label{dev:wp4r3}
    \wpdeliverable[13]{georgio}{P}{PU}{Merging code of Models 1 and 2 to master.}\label{dev:wp4merge2}
    \wpdeliverable[18]{georgio}{D}{PU}{A demonstration of the ability to play a game of 2P Super Mario Bros using cogniLink.}\label{dev:wp4vid}
    \wpdeliverable[18]{georgio}{R}{PU}{Main Report update with full progress accomplished after the end of WP4.}\label{dev:MR0.4}
    
  \end{wpdeliverables}
\end{workpackage}
%%% Local Variables:
%%% mode: latex
%%% TeX-master: "proposal-main"
%%% End:

%%%%%%%%%%%%%%%%%%%%%%%%%%%%%%
%  Work Package Description  %
%%%%%%%%%%%%%%%%%%%%%%%%%%%%%%

\begin{workpackage}{Optimizations}
  \label{wp:opt} %change and use appropriate description

  %%%%%%%%%%%%%%%%%% TOP TABLE %%%%%%%%%%%%%%%%%%%%%%%%%%%%%
  % Data for the top table
  \wpstart{18} %Starting Week
  \wpend{20} %End Week
  \wptype{Activity type} %RTD, DEM, MGT, or OTHER

  % Person Weeks per participant (required, max 7, * for leader)  
  % syntax: \personweeks{Participant number}{value}    (not wp leader)
  %     or  \personweeks{Participant short name}{value} (not wp leader)
  %         \personweeks*{Participant number}{value}    (wp leader)
  % for example:
  \personweeks*{Georgio}{2}
  % etc.

  \makewptable % Work package summary table
    
  % Work Package Objectives
  \begin{wpobjectives}
    This work package has the following objectives:
    \begin{enumerate}
 \item Optimizing the code in such a way that a trigger thought is recognized 
in realtime.
    \end{enumerate}
  \end{wpobjectives}
  
  % Work Package Description
  \begin{wpdescription}
    % Divide work package into multiple tasks.
    % Use \wptask command
    % syntax: \wptask{leader}{contributors}{start-m}{end-m}{title}{description}   
 
    \wptask{Georgio}{Georgio}{18}{20}{Task1}{
      \label{task:wp1task1}
      Code optimization with the main goal of reducing latency.
    }

    
  \end{wpdescription}
  
  % Work Package Deliverable
  \begin{wpdeliverables}
    % Data for the deliverables and milestones  tables
    % syntax: \deliverable[delivery date]{nature}{dissemination
    % level}{description} 
    %
    % nature: R = Report, P = Prototype, D = Demonstrator, O = Other
    % dissemination level: PU = Public, PP = Restricted to other
    % programme participants (including the Commission Services), RE =
    % Restricted to a group specified by the consortium (including the
    % Commission Services), CO = Confidential, only for members of the
    % consortium (including the Commission Services).
    % 
    % \wpdeliverable[date]{R}{PU}{A report on \ldots}

    \wpdeliverable[20]{Georgio}{P}{PU}{Optimized code will be merged to master.}\label{dev:wp5merge}

    \wpdeliverable[20]{Georgio}{R}{PU}{Main Report update with full progress accomplished after the end of WP5, emphasizing on measures taken for optimizing code.}\label{dev:MR0.4}
    
  \end{wpdeliverables}
\end{workpackage}
%%% Local Variables:
%%% mode: latex
%%% TeX-master: "proposal-main"
%%% End:

%%%%%%%%%%%%%%%%%%%%%%%%%%%%%%
%  Work Package Description  %
%%%%%%%%%%%%%%%%%%%%%%%%%%%%%%

\begin{workpackage}{Real Life Application}
  \label{wp:app} %change and use appropriate description


  %%%%%%%%%%%%%%%%%% TOP TABLE %%%%%%%%%%%%%%%%%%%%%%%%%%%%%
  % Data for the top table
  \wpstart{20} %Starting Week
  \wpend{30} %End Week
  \wptype{Activity type} %RTD, DEM, MGT, or OTHER

  % Person Weeks per participant (required, max 7, * for leader)  
  % syntax: \personweeks{Participant number}{value}    (not wp leader)
  %     or  \personweeks{Participant short name}{value} (not wp leader)
  %         \personweeks*{Participant number}{value}    (wp leader)
  % for example:
  \personweeks*{georgio}{10}
  % etc.

  \makewptable % Work package summary table
    
  % Work Package Objectives
  \begin{wpobjectives}
    This work package has the following objectives:
    \begin{enumerate}
    \item Implement cogniLink to work on a controller of a wheelchair, this task derives from cogniLink as a forked project;
    \item Initiate research about Locked-in syndrome: Find suitiable use-cases and patients.
    \end{enumerate}
  \end{wpobjectives}

  % Work Package Description
  \begin{wpdescription}
    % Divide work package into multiple tasks.
    % Use \wptask command
    % syntax: \wptask{leader}{contributors}{start-m}{end-m}{title}{description}   
 
    \wptask{georgio}{georgio}{20}{22}{Task1}{
      \label{task:wp6task1}
      First Fork of cogniLink.
      A wheelchair controller/helper that can be fed command data as a replacement to the virtual HID will be designed and implemented.
    }
    \wptask{georgio}{georgio}{20}{30}{Task2}{
      \label{task:wp6task2}
      Initiation of formal research about Locked-in syndrome, patients, and practical use-case scenarios.
    }  
  \end{wpdescription}
  
  % Work Package Deliverable
  \begin{wpdeliverables}
    % Data for the deliverables and milestones  tables
    % syntax: \deliverable[delivery date]{nature}{dissemination
    % level}{description} 
    %
    % nature: R = Report, P = Prototype, D = Demonstrator, O = Other
    % dissemination level: PU = Public, PP = Restricted to other
    % programme participants (including the Commission Services), RE =
    % Restricted to a group specified by the consortium (including the
    % Commission Services), CO = Confidential, only for members of the
    % consortium (including the Commission Services).
    % 
    % \wpdeliverable[date]{R}{PU}{A report on \ldots}

    \wpdeliverable[22]{georgio}{R}{PU}{Report about integration with wheelchair.}\label{dev:wp6f1report}
    
    \wpdeliverable[30]{georgio}{P}{PU}{Forking cogniLink and merging changes needed for wheelchair integration.}\label{dev:wp6fork}

    \wpdeliverable[30]{georgio}{R}{PU}{Report about research findings and future steps towards integrating with a locked in patient. }\label{dev:wp6r2research}

    \wpdeliverable[30]{georgio}{R}{PU}{Main Main Report update with full progress accomplished after the end of WP6.}\label{dev:MR0.6}
  
  \end{wpdeliverables}
\end{workpackage}
%%% Local Variables:
%%% mode: latex
%%% TeX-master: "proposal-main"
%%% End:

%%%%%%%%%%%%%%%%%%%%%%%%%%%%%%
%  Work Package Description  %
%%%%%%%%%%%%%%%%%%%%%%%%%%%%%%

\begin{workpackage}{Universal Model, n Commands}
  \label{wp:mUcN} %change and use appropriate description

  %%%%%%%%%%%%%%%%%% TOP TABLE %%%%%%%%%%%%%%%%%%%%%%%%%%%%%
  % Data for the top table
  \wpstart{30} %Starting Week
  \wpend{40} %End Week
  \wptype{Activity type} %RTD, DEM, MGT, or OTHER

  % Person Weeks per participant (required, max 7, * for leader)  
  % syntax: \personweeks{Participant number}{value}    (not wp leader)
  %     or  \personweeks{Participant short name}{value} (not wp leader)
  %         \personweeks*{Participant number}{value}    (wp leader)
  % for example:
  \personweeks*{georgio}{10}
  % etc.

  \makewptable % Work package summary table
    
  % Work Package Objectives
  \begin{wpobjectives}
    This work package has the following objectives:
    \begin{enumerate}
      \item Integrating cogniLink with a Locked-in syndrome patient, in such a way for them to be able to mentally execute a command;
      \item Designing, implementing, and training a universal model (UM).
    \end{enumerate}
  \end{wpobjectives}
  
  % Work Package Description
  \begin{wpdescription}
    % Divide work package into multiple tasks.
    % Use \wptask command
    % syntax: \wptask{leader}{contributors}{start-m}{end-m}{title}{description}   
 
    \wptask{georgio}{georgio}{30}{40}{Task1}{
      \label{task:wp7task1}
      Second fork of cogniLink: Training a new model with a Locked-in patient.
    }
    \wptask{georgio}{georgio}{30}{40}{Task2}{
      \label{task:wp7task2}
      Training UM using multiple models trained on multiple commands.
    }    
    \wptask{georgio}{georgio}{30}{40}{Task3}{
      Code optimization.
    }

    
  \end{wpdescription}
  
  % Work Package Deliverable
\begin{wpdeliverables}
    % Data for the deliverables and milestones  tables
    % syntax: \deliverable[delivery date]{nature}{dissemination
    % level}{description} 
    %
    % nature: R = Report, P = Prototype, D = Demonstrator, O = Other
    % dissemination level: PU = Public, PP = Restricted to other
    % programme participants (including the Commission Services), RE =
    % Restricted to a group specified by the consortium (including the
    % Commission Services), CO = Confidential, only for members of the
    % consortium (including the Commission Services).
    % 
    % \wpdeliverable[date]{R}{PU}{A report on \ldots}

    \wpdeliverable[40]{georgio}{R}{PU}{Final Main Report update on Universal Model, Locked-in patient progress, and future plans for cogniLink.}\label{dev:MR1.0}
    \wpdeliverable[40]{georgio}{P}{CO}{Publishing Universal Model.}\label{dev:wp7um}
    \wpdeliverable[40]{georgio}{P}{CO}{Publishing model trained on Locked-in patient.}\label{dev:wp7limodel}
    \wpdeliverable[40]{georgio}{P}{PU}{Merging optimized code to master.}\label{dev:wp7merge}
  \end{wpdeliverables}
\end{workpackage}
%%% Local Variables:
%%% mode: latex
%%% TeX-master: "proposal-main"
%%% End:


\subsection{List of work packages}
\label{sec:wplist}
\makewplist

\subsection{List of deliverables}\footnote{If your action taking part in the Pilot on Open Research Data, you must include a data management plan as a distinct deliverable within the first 6 weeks of the project.  This deliverable will evolve during the lifetime of the project in order to present the status of the project's reflections on data management. A template for such a plan is available on the Participant Portal (Guide on Data Management).}
\label{sec:deliverables}
\instructions{
\textbf{KEY}\\
\emph{Deliverable numbers in order of delivery dates. Please use the numbering convention $<$WP number$>.<$number of deliverable within that WP$>$.}
\vskip0.2cm
\noindent\emph{For example, deliverable 4.2 would be the second deliverable from work package 4.}
\vskip0.2cm
\noindent\textbf{Type:}\\
\emph{Use one of the following codes:}\\
\indent R: Document, report (excluding the periodic and final reports)\\
\indent DEM: Demonstrator, pilot, prototype, plan designs\\
\indent DEC: Websites, patents filing, press \& media actions, videos, etc.\\
\indent OTHER: Software, technical diagram, etc.
\vskip0.2cm
\noindent\textbf{Dissemination level}:\\
\emph{Use one of the following codes:}\\
\indent PU = Public, fully open, e.g. web\\
\indent CO = Confidential, restricted under conditions set out in Model Grant Agreement\\
\indent CI = Classified, information as referred to in Commission Decision 2001/844/EC.\\
\vskip0.2cm
\noindent\textbf{Delivery date}:\\
Measured in weeks from the project start date (week 1).
}
\makedeliverablelist


\section{Management and risk assessment}
\label{sec:management}
\setcounter{table}{0}
\instructions{
\begin{itemize}
\item Describe the organisational structure and the decision-making ( including a list of
milestones (table 3.2a))
\item Describe any critical risks, relating to project implementation, that the stated project's objectives may not be achieved. Detail any risk mitigation measures. Please provide a table with critical risks identified and mitigating actions (table 3.2b)
\end{itemize}
}



\subsection{List of milestones}
\label{sec:milestones}
\instructions{
\vskip0.2cm
\noindent\textbf{KEY}\\
\textbf{Estimated date}\\
\emph{Measured in weeks from the project start date (week 1)}
\vskip0.2cm
\noindent\textbf{Means of verification}\\
\emph{Show how you will confirm that the milestone has been attained. Refer to indicators if appropriate. For example: a laboratory prototype that is ‘up and running’; software released and validated by a user group; field survey complete and data quality validated.}}

\milestone[24]{Completed simulator development}{Software released and
  validated}{1}
\milestone[36]{Final demonstration}{Application of results}{WP\,\ref{wp:test}}

\makemilestoneslist

\subsection{Critical risks for implementation}
\label{sec:risks}

\criticalrisk{The dedicated chip sent to fabrication is not functional.}{WP\,\ref{wp:test}}{Resort to Software simulations}

\makerisklist

\section{Consortium as a whole} 
\label{sec:consortium}
\instructions{
\emph{The individual members of the consortium are described in a separate section 4. There is no need to repeat that information here.}
\begin{itemize}
\item Describe the consortium. How will it match the project’s objectives? How do the members complement one another (and cover the value chain, where appropriate)? In what way does each of them contribute to the project? How will they be able to work effectively together? 
\item If applicable, describe how the project benefits from any industrial/SME involvement.
\item \textbf{Other countries:} If one or more of the participants requesting EU funding is based in a country that is not automatically eligible for such funding (entities from Member States of the EU, from Associated Countries and from one of the countries in the exhaustive list included in General Annex A of the work programme are automatically eligible for EU funding), explain why the participation of the entity in question is essential to carrying out the project. 
\end{itemize}
}

\section{Resources to be committed} 
\label{sec:resources}
\setcounter{table}{0}
\instructions{
\emph{Please make sure the information in this section matches the costs as stated in the budget table in section 3 of the administrative proposal forms, and the number of person/weeks, shown in the detailed work package descriptions.}
\vskip0.2cm
Please provide the following:\\
\begin{itemize}
\item a table showing number of person/weeks required (table 3.4a)
\item a table showing ``other direct costs'' (table 3.4b) for participants where those costs exceed 15\% of the personnel costs (according to the budget table in section 3 of the administrative proposal forms)  
\end{itemize}
}

\subsection{Summary of staff efforts}
\instructions{Table 3.4a: \emph{Please indicate the number of person/weeks over the whole duration of the planned work, for each work package, for each participant. Identify the work-package leader for each WP by showing the relevant person-week figure in bold.}}
\makesummaryofefforttable

\subsection{‘Other direct cost’ items (travel, equipment, other goods and services, large research infrastructure)}
\instructions{
Please provide a table of summary of costs for each participant, if the sum of the costs for ``travel'', ``equipment'', and ``goods and services'' exceeds 15\% of the personnel costs for that participant (according to the budget table in section 3 of the proposal administrative forms).
}

\costsTravel{georgio}{2500}{3 pairwise meetings for 2 people, 2 conferences for 3 people, 3 internal project meetings for 3 people}
\costsEquipment{georgio}{3000}{CAD workstation for chip design}
\costsOther{georgio}{60000}{Fabrication of 2 VLSI chips}

\makecoststable


\instructions{
Please complete the table below for all participants that would like to declare costs of large research infrastructure under Article 6.2 of the General Model Agreement, irrespective of the percentage of personnel costs. Please indicate (in the justification) if the beneficiary's methodology for declaring the costs for large research infrastructure has already been positively assessed by the Commission.\\
Note: Large research infrastructure means research infrastructure of a total value of at least EUR 20 million, for a beneficiary. More information and further guidance on the direct costing for the large research infrastructure is available in the H2020 Online Manual on the Participant Portal.
}

\costslri{georgio}{400000}{Synchrotron}

\makelritable
