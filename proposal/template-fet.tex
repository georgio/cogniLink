%% EU FET Open  Proposal LaTeX template
%% V1.0
%% Based on the h2020proposal.cls LaTeX class for writing EU H2020 RIA proposals.
%% 
%% Copyright (c) 2010, Giacomo Indiveri
%%
%%  This latex class is free software: you can redistribute it and/or modify
%%  it under the terms of the GNU General Public License as published by
%%  the Free Software Foundation, either version 3 of the License, or
%%  (at your option) any later version.
%%
%%  h2020proposal.cls is distributed in the hope that it will be useful,
%%  but WITHOUT ANY WARRANTY; without even the implied warranty of
%%  MERCHANTABILITY or FITNESS FOR A PARTICULAR PURPOSE.  See the
%%  GNU General Public License for more details.
%%
%%  You should have received a copy of the GNU General Public License
%%  along with h2020proposal.cls.  If not, see <http://www.gnu.org/licenses/>.
%%
%% Contributors: Elisabetta Chicca
%%
%% Disclaimer: The template is based on the document provided by the EU Participants Portal 
%% "Part B  Template FETOPEN sections 1-3 Final.doc"
%%
%% Use the original source and the http://ec.europa.eu/ documentation for reference. We make no
%% representations or warranties of any kind, express or implied, about the completeness, accuracy,
%% reliability, suitability or availability with respect to the original template.
%% In no event will we be liable for any loss or damage including without limitation, indirect or
%% consequential loss or damage, or any loss or damage whatsoever arising out of, or in connection
%% with, the use of this template and/or class.
%%
%% Makes use of the memoir class. Read the optimum memman documentation for
%% info on how to customize your proposal.


\documentclass[]{h2020proposal}     % Remove 'draft' option for final version
%\documentclass[draft]{h2020proposal} % Use 'draft' option to show comments and labels

% For in-line comments use:
% \marginpar{comment text}

%% Extra Packages
%% ========
%\usepackage{fontspec}% Latin Modern by default with xelatex

%% LaTeX Font encoding -- DO NOT CHANGE
\usepackage[OT1]{fontenc}

%% Input encoding 'utf8'. In some cases you might need 'utf8x' for
%% extra symbols. Not all editors, especially on Windows, are UTF-8
%% capable, so you may want to use 'latin1' instead.
%\usepackage[utf8,latin1]{inputenc}

%% Babel provides support for languages.  'english' uses British
%% English hyphenation and text snippets like "Figure" and
%% "Theorem". Use the option 'ngerman' if your document is in German.
%% Use 'american' for American English.  Note that if you change this,
%% the next LaTeX run may show spurious errors.  Simply run it again.
%% If they persist, remove the .aux file and try again.
\usepackage[english]{babel}

%% For underlined wrapped text.
\usepackage{soul}

%% This changes default fonts for both text and math mode to use Herman Zapfs
%% excellent Palatino font.  Do not change this.
\usepackage[sc]{mathpazo} % Not needed with xelatex

%% The AMS-LaTeX extensions for mathematical typesetting.  Do not
%% remove.
\usepackage{amsmath,amssymb,amsfonts,mathrsfs}

%%--Refs apa style package
\usepackage{apacite}

%% Gantt Charts in LaTeX
\usepackage{pgfgantt}

%% LaTeX' own graphics handling
\usepackage{graphicx}

%% Fancy character protrusion.  Must be loaded after all fonts.
\usepackage[activate]{pdfcprot}

%% Nicer tables.  Read the excellent documentation.
\usepackage{booktabs}

% Compressed itemized lists (with a * at the end)
\usepackage{mdwlist}

%% Nicer URLs.  
\usepackage{url}

%% Configure citation styles
\usepackage[numbers,sort&compress,square]{natbib}
\def\bibfont{\footnotesize}     %for smaller fonts in the biblio section

%% Hyper Ref package. In order to operate correctly, it must be the last package declared
\usepackage[colorlinks,pagebackref,breaklinks]{hyperref} 

%% Extra package options
\hypersetup{
  hypertexnames=true, linkcolor=blue, anchorcolor=black,
  citecolor=blue, urlcolor=blue  
}

\urlstyle{rm} %so it doesn't use a typewriter font for urls.
\DeclareGraphicsExtensions{.jpg,.pdf,.mps,.png} % for pdflatex
\graphicspath{{img/} {./}} %put all figures in these dirs

\newcommand{\alert}[1]{{\color{red}\textbf{#1}}}

%%%%%%%%%%%%%%%%%%%%%%%%%%%%%%%%%%%%%%%%%%%%%%%%%%%%%%%%%%%%%%%%%%%%%%

%% ========================================================
%% IMPORTANT store proposal information in global variables
%% ========================================================
\title{cogniLink$:$ A Non-Invasive Brain-Computer Interface 
That Enables Seamless Execution of Commands Through Thought 
Recognition}
\shortname{Project cogniLink} 
\titlelogo{}{0.25} % file name and scale
\fundingscheme{The Nadim Kobeissi Fund}
\topic{Work Programme topic addressed}
\coordinator{Georgio Nicolas}{me@georgio.xyz}{+1 (929) 250-4490}
\participant{University of Coordinator}{UoC}{Lebanon} % First participant is the coordinator
\participant{University of partner 2}{UoP2}{Country2} % as example...
\participant{University of partner 3}{UoP3}{Country3} % as example...
% etc.

% Page Headers
%\makeoddhead{proposal}{\disptoken{@acronym}}{}{\rightmark}
%\makeevenhead{proposal}{\leftmark}{}{\disptoken{@acronym}}

%Page Footers
%\makeevenfoot{proposal}{ \thepage }{ \date{\today} }{ \disptoken{@acronym} }
%\makeoddfoot{proposal}{  }{ \date{\today} }{ \thepage }

%Page Style
\pagestyle{proposal} %use \pagestyle{showlocs} for debugging

%Heading style 
\makeheadstyles{default}{%
\renewcommand*{\chapnamefont}{\normalfont\bfseries}
\renewcommand*{\chapnumfont}{\normalfont\bfseries}
\renewcommand*{\chaptitlefont}{\normalfont\bfseries}
\renewcommand*{\secheadstyle}{\normalfont\bfseries}
}%
\headstyles{default}

%Chapter Style
\chapterstyle{section} %Avoid writing the word "Chapter" at the beginning of each proposal section
% other possible valid styles:
% article, bringhurst, crosshead, culver, dash, demo2, ell, southall, tandh, verville, wilsondob
\renewcommand*{\chaptitlefont}{\normalfont\Large\bfseries}
\renewcommand*{\chapnumfont}{\normalfont\Large\bfseries}

\begin{document}
%% TITLE
\maketitle
\vspace{-4em}
\renewcommand\contentsname{\normalsize Table of Contents \vspace{-4em}}
\setlength{\cftbeforechapterskip}{1.0em plus 0.3em minus 0.1em}
\renewcommand{\cftchapterbreak}{\addpenalty{-4000}}
%\makeparticipantstable          %for the ICT RIA proposals
\begin{abstract}
Although great strides have been achieved in making 
computers more accessible, it’s indisputable that 
there remains huge prospects for improvement. Given that 
technology is designed for the masses, it offers every 
individual a platform to do what is needed; this 
includes, but isn’t limited to, support for 
individuals with motor, dexterity, and/or speech 
impairments.  In this proposal, we will discuss 
cogniLink, a tool that assists developers in making 
computers more accessible for persons with afflictions. 
cogniLink is a brain-computer interface that allows the 
user to trigger the execution of a command simply by 
thinking of the trigger. A training data set is to be 
collected from n-users in order to train n-models using 
an ElectroEncephaloGram (EEG). Each model is programmed 
to recognize one or more trigger thoughts. The same model 
interacts with a stack of software which allows it to map 
positive outputs from the model and transform it into an 
actionable command. For the purpose of demonstration, the 
model will be trained to recognize commands from one user 
which will be mapped to a virtualHID in such a way that 
allows the user to play Super Mario Bros. After an 
extensive process of training n-models, a universal model 
(UM) will be trained using data from the aforementioned 
n-models in order to have a simpler training process for 
new users. cogniLink will allow disabled people to 
execute commands in a very seamless and orderly fashion, 
thus making computers more accessible to persons with 
digital input impairments. Two of cogniLink’s long 
term goals are to allow an amputee to be able to 
effortlessly be able to control a wheelchair in real 
time, and for someone suffering from Locked-In Syndrome 
to be able to interact with the world around them with 
ease.
\end{abstract}

\thispagestyle{empty}
\pagebreak

%% Main proposal

%% Fixed proposal structure - Do not change
\chapter{Implementation}
\label{cha:implementation}

\section{Project work plan}
\label{sec:work-plan}
\instructions{
Please provide the following:
\begin{itemize}
\item brief presentation of the overall structure of the work plan;
\item timing of the different work packages and their components (Gantt chart or similar);
\item detailed work description, i.e.:
\begin{itemize}
\item a description of each work package (table 3.1a);
\item a list of work packages (table 3.1b);
\item a list of major deliverables (table 3.1c);
\end{itemize}
\item graphical presentation of the components showing how they inter-relate (Pert chart or similar).
\end{itemize}
\vskip0.2cm
\emph{\indent Give full details. Base your account on the logical structure of the project and the stages in which it is to be carried out. Include details of the resources to be allocated to each work package. The number of work packages should be proportionate to the scale and complexity of the project.}
\vskip0.2cm
\emph{You should give enough detail in each work package to justify the proposed resources to be allocated and also quantified information so that progress can be monitored, including by the Commission.}
\vskip0.2cm
\emph{You are advised to include a distinct work package on ``management'' (see section 3.2) and to give due visibility in the work plan to ``dissemination and exploitation'' and ``communication activities'', either with distinct tasks or distinct work packages.}
\vskip0.2cm
\emph{You will be required to include an updated (or confirmed) ``plan for the dissemination and exploitation of results'' in both the periodic and final reports. (This does not apply to topics where a draft plan was not required.) This should include a record of activities related to dissemination and exploitation that have been undertaken and those still planned. A report of completed and planned communication activities will also be required.}
\vskip0.2cm
\emph{If your project is taking part in the Pilot on Open Research Data\footnote{Certain actions under Horizon 2020 participate in the ‘Pilot on Open Research Data in Horizon 2020’. All other actions can participate on a voluntary basis to this pilot.  Further guidance is available in the H2020 Online Manual on the Participant Portal.}, you must include a 'data management plan' as a distinct deliverable within the first 6 weeks of the project. A template for such a plan is given in the guidelines on data management in the H2020 Online Manual. This deliverable will evolve during the lifetime of the project in order to present the status of the project's reflections on data management.}
\vskip0.2cm
\emph{\noindent \textbf{Definitions:}}
\vskip0.2cm
\emph{\ul{``Work package''} means a major sub-division of the proposed project.}
\vskip0.2cm
\emph{\ul{``Deliverable''} means a distinct output of the project, meaningful in terms of the project's overall objectives and constituted by a report, a document, a technical diagram, a software etc.}
\vskip0.2cm
\emph{\ul{``Milestones''} means control points in the project that help to chart progress. Milestones may correspond to the completion of a key deliverable, allowing the next phase of the work to begin. They may also be needed at intermediary points so that, if problems have arisen, corrective measures can be taken. A milestone may be a critical decision point in the project where, for example, the consortium must decide which of several technologies to adopt for further development.}
\vskip0.2cm
\emph{\noindent Report on work progress is done primarily through the periodic and final reports. Deliverables should complement these reports and should be kept to the minimum necessary.}
}

% Gantt chart in latex (requires pfggantt.sty file)
%%project Gantt chart


\begin{figure}
  \centering

  \begin{ganttchart}%
    [
    x unit = 0.25cm,
    y unit title=0.4cm,
    y unit chart=0.4cm,
    vgrid,
    title/.style={draw=black!50, fill=green!50!black},
    title label font=\sffamily\bfseries\color{white},
    title label anchor/.style={below=-1.6ex},
    title left shift=.05,
    title right shift=-.05,
    title height=1,
    bar/.style={draw=none, fill=blue!75},
    bar height=.6,
    bar label font=\small\color{black!50},
    milestone label font=\small\color{red!50},
    group right shift=0,
    group top shift=.6,
    group height=.3,
    group peaks={}{}{.2},
    incomplete/.style={fill=red}]{60}
    
    \gantttitle{ACRONYM}{60} \\
    \gantttitle{Year 1}{12} 
    \gantttitle{Year 2}{12} 
    \gantttitle{Year 3}{12}  
    \gantttitle{Year 4}{12}  
    \gantttitle{Year 5}{12} \\ 
    
    \ganttchartdata % data generated by the ICTProposal.cls
    
  \end{ganttchart}

  \caption[Gantt chart]{Project Gantt chart.}
  \label{fig:gantt}
\end{figure}


\subsection{Work package description}
\label{sec:wps}

%Include work-packages as separate files
%%%%%%%%%%%%%%%%%%%%%%%%%%%%%%
%  Work Package Description  %
%%%%%%%%%%%%%%%%%%%%%%%%%%%%%%

\begin{workpackage}{Virtual HID, Data Collection, and ML Code}
  \label{wp:data} %change and use appropriate description

  %%%%%%%%%%%%%%%%%% TOP TABLE %%%%%%%%%%%%%%%%%%%%%%%%%%%%%
  % Data for the top table
  \wpstart{1} %Starting Week
  \wpend{36} %End Week
  \wptype{Activity type} %RTD, DEM, MGT, or OTHER

  % Person Weeks per participant (required, max 7, * for leader)  
  % syntax: \personweeks{Participant number}{value}    (not wp leader)
  %     or  \personweeks{Participant short name}{value} (not wp leader)
  %         \personweeks*{Participant number}{value}    (wp leader)
  % for example:
  \personweeks*{georgio}{12}
  % etc.

  \makewptable % Work package summary table
    
  % Work Package Objectives
  \begin{wpobjectives}
    This work package has the following objectives:
    \begin{enumerate}
    \item To develop a Virtual Human Interface Device;
    \item To develop an  API that gathers raw data from the Cyton board and feeds it 
to a CSV file;
    \item To write code needed to efficiently store and manage datasets;
    \item To write code needed to start training Model 1 on Command A.
    \end{enumerate}
  \end{wpobjectives}
  
  % Work Package Description
  \begin{wpdescription}
    % Divide work package into multiple tasks.
    % Use \wptask command
    % syntax: \wptask{leader}{contributors}{start-m}{end-m}{title}{description}   
 
    \wptask{georgio}{georgio}{1}{12}{Task1}{
      \label{task:wp1task1}
      Here we will test the WP Task code. 
    }
    \wptask{georgio}{georgio}{6}{9}{Task2}{
      \label{task:wp2task2}
      In this task UZH will integrate the work done in ~\ref{task:wp1test}.
    }    
    \wptask{georgio}{All other}{9}{12}{Task3}{
      Here all the WP participants will apply the results to...

    }

    
  \end{wpdescription}
  
  % Work Package Deliverable
  \begin{wpdeliverables}
    % Data for the deliverables and milestones  tables
    % syntax: \deliverable[delivery date]{nature}{dissemination
    % level}{description} 
    %
    % nature: R = Report, P = Prototype, D = Demonstrator, O = Other
    % dissemination level: PU = Public, PP = Restricted to other
    % programme participants (including the Commission Services), RE =
    % Restricted to a group specified by the consortium (including the
    % Commission Services), CO = Confidential, only for members of the
    % consortium (including the Commission Services).
    % 
    % \wpdeliverable[date]{R}{PU}{A report on \ldots}

    \wpdeliverable[36]{georgio}{R}{PU}{Report on the definition of the model
      specifications.}\label{dev:wp1specs}
    
    \wpdeliverable[12]{georgio}{R}{PU}{Report on Feasibility study for the model
      implementation.}\label{dev:wp1implementation}

    \wpdeliverable[24]{georgio}{R}{PU}{Prototype of model
      implementation.}\label{dev:wp1prototype}

  \end{wpdeliverables}
\end{workpackage}
%%% Local Variables:
%%% mode: latex
%%% TeX-master: "proposal-main"
%%% End:
             %Use \input for first WP
%%%%%%%%%%%%%%%%%%%%%%%%%%%%%%
%  Work Package Description  %
%%%%%%%%%%%%%%%%%%%%%%%%%%%%%%

\begin{workpackage}{Model 1 Command A}
  \label{wp:m1cA} %change and use appropriate description

  %%%%%%%%%%%%%%%%%% TOP TABLE %%%%%%%%%%%%%%%%%%%%%%%%%%%%%
  % Data for the top table
  \wpstart{2} %Starting Week
  \wpend{36} %End Week
  \wptype{Activity type} %RTD, DEM, MGT, or OTHER

  % Person Weeks per participant (required, max 7, * for leader)  
  % syntax: \personweeks{Participant number}{value}    (not wp leader)
  %     or  \personweeks{Participant short name}{value} (not wp leader)
  %         \personweeks*{Participant number}{value}    (wp leader)
  % for example:
  \personweeks*{georgio}{3}  % etc.

  \makewptable % Work package summary table
    
  % Work Package Objectives
  \begin{wpobjectives}
    \begin{enumerate}
    \item To collect training, validation, and test datasets for 
Model 1 Command A; 
    \item Training Model 1 using aforementioned data;
    \item Testing/Patching Model 1.
    \end{enumerate}
  \end{wpobjectives}
  
  % Work Package Description
  \begin{wpdescription}
    % Divide work package into multiple tasks.
    % Use \wptask command
    % syntax: \wptask{leader}{contributors}{start-m}{end-m}{title}{description}   
 
    Description of work carried out in WP, broken down into tasks, and
    with role of partners list. Use the \texttt{\textbackslash wptask} command.

    \wptask{georgio}{georgio}{1}{12}{Task1}{
      \label{task:wp2task1}
      Here we will test the WP Task code. 
    }
    \wptask{georgio}{georgio}{6}{9}{Task2}{
      \label{task:wp2task2}
      In this task UZH will integrate the work done in ~\ref{task:wp2test}.
    }    
    \wptask{georgio}{All other}{9}{12}{Task 3}{
      Here all the WP participants will apply the results to...
    }
    
   
  \end{wpdescription}
  
  % Work Package Deliverable
  \begin{wpdeliverables}
    % Data for the deliverables and milestones  tables
    % syntax: \deliverable[delivery date]{nature}{dissemination
    % level}{description} 
    %
    % nature: R = Report, P = Prototype, D = Demonstrator, O = Other
    % dissemination level: PU = Public, PP = Restricted to other
    % programme participants (including the Commission Services), RE =
    % Restricted to a group specified by the consortium (including the
    % Commission Services), CO = Confidential, only for members of the
    % consortium (including the Commission Services).
    % 
    % \wpdeliverable[date]{R}{PU}{A report on \ldots}

    \wpdeliverable[36]{georgio}{R}{PU}{Report on the definition of the model
      specifications.}\label{dev:wp2specs}
    
    \wpdeliverable[12]{georgio}{R}{PU}{Report on Feasibility study for the model
      implementation.}\label{dev:wp2implementation}

    \wpdeliverable[24]{georgio}{R}{PU}{Prototype of model
      implementation.}\label{dev:wp2prototype}

  \end{wpdeliverables}

\end{workpackage}


%%% Local Variables:
%%% mode: latex
%%% TeX-master: "proposal-main"
%%% End:

%%%%%%%%%%%%%%%%%%%%%%%%%%%%%%
%  Work Package Description  %
%%%%%%%%%%%%%%%%%%%%%%%%%%%%%%

\begin{workpackage}{Model 1 n Commands}
  \label{wp:m1cN} %change and use appropriate 
description

  %%%%%%%%%%%%%%%%%% TOP TABLE %%%%%%%%%%%%%%%%%%%%%%%%%%%%%
  % Data for the top table
  \wpstart{1} %Starting Week
  \wpend{36} %End Week
  \wptype{Activity type} %RTD, DEM, MGT, or OTHER

  % Person Weeks per participant (required, max 7, * for leader)  
  % syntax: \personweeks{Participant number}{value}    (not wp leader)
  %     or  \personweeks{Participant short name}{value} (not wp leader)
  %         \personweeks*{Participant number}{value}    (wp leader)
  % for example:
  \personweeks*{UoP3}{12}
  % etc.

  \makewptable % Work package summary table
    
  % Work Package Objectives
  \begin{wpobjectives}
    This work package has the following objectives:
    \begin{enumerate*}
    \item Link the output from the Model to the virtual HID created in WP1;
    \item Map trigger thoughts to button presses; 
    \item Play a game of 1P Super Mario Bros.
    \end{enumerate*}
  \end{wpobjectives}
  
  % Work Package Description
  \begin{wpdescription}
    % Divide work package into multiple tasks.
    % Use \wptask command
    % syntax: \wptask{leader}{contributors}{start-m}{end-m}{title}{description}   
 
    Description of work carried out in WP, broken down into tasks, and
    with role of partners list. Use the \texttt{\textbackslash wptask} command.

    \wptask{UoC}{UoC}{1}{12}{Test}{
      \label{task:wp3test}
      Here we will test the WP Task code. 
    }
    \wptask{UoC}{UoC}{6}{9}{Integrate}{
      \label{task:wp3integrate}
      In this task UZH will integrate the work done in ~\ref{task:wp3test}.
    }    
    \wptask{UoP3}{All other}{9}{12}{Apply}{
      Here all the WP participants will apply the results to...
    }
    
    \paragraph{Role of partners}
    \begin{description*}
    \item[Participant short name] will lead Task~\ref{task:wp3integrate}.
    \item[UoC] will..
    \end{description*}
  \end{wpdescription}
  
  % Work Package Deliverable
  \begin{wpdeliverables}
    % Data for the deliverables and milestones  tables
    % syntax: \deliverable[delivery date]{nature}{dissemination
    % level}{description} 
    %
    % nature: R = Report, P = Prototype, D = Demonstrator, O = Other
    % dissemination level: PU = Public, PP = Restricted to other
    % programme participants (including the Commission Services), RE =
    % Restricted to a group specified by the consortium (including the
    % Commission Services), CO = Confidential, only for members of the
    % consortium (including the Commission Services).
    % 
    % \wpdeliverable[date]{R}{PU}{A report on \ldots}

    \wpdeliverable[36]{UoC}{R}{PU}{Report on the definition of the model
      specifications.}\label{dev:wp3specs}
    
    \wpdeliverable[12]{UoP3}{R}{PU}{Report on Feasibility study for the model
      implementation.}\label{dev:wp3implementation}

    \wpdeliverable[24]{UoP2}{R}{PU}{Prototype of model
      implementation.}\label{dev:wp3prototype}

  \end{wpdeliverables}

\end{workpackage}


%%% Local Variables:
%%% mode: latex
%%% TeX-master: "proposal-main"
%%% End:

%%%%%%%%%%%%%%%%%%%%%%%%%%%%%%
%  Work Package Description  %
%%%%%%%%%%%%%%%%%%%%%%%%%%%%%%

\begin{workpackage}{Model 2 n Commands}
  \label{wp:m2cN} %change and use appropriate 
description

  %%%%%%%%%%%%%%%%%% TOP TABLE %%%%%%%%%%%%%%%%%%%%%%%%%%%%%
  % Data for the top table
  \wpstart{1} %Starting Week
  \wpend{36} %End Week
  \wptype{Activity type} %RTD, DEM, MGT, or OTHER

  % Person Weeks per participant (required, max 7, * for leader)  
  % syntax: \personweeks{Participant number}{value}    (not wp leader)
  %     or  \personweeks{Participant short name}{value} (not wp leader)
  %         \personweeks*{Participant number}{value}    (wp leader)
  % for example:
  \personweeks*{UoC}{12}
  % etc.

  \makewptable % Work package summary table
    
  % Work Package Objectives
  \begin{wpobjectives}
    This work package has the following objectives:
    \begin{enumerate}
    \item Replicate all steps in WP2 and WP3 so that we get a Model 2 for a different individual trained on n the same n Commands as Model 1;
    \item Play a game of 2P Super Mario Bros.
    \end{enumerate}
  \end{wpobjectives}
  
  % Work Package Description
  \begin{wpdescription}
    % Divide work package into multiple tasks.
    % Use \wptask command
    % syntax: \wptask{leader}{contributors}{start-m}{end-m}{title}{description}   
 
    \wptask{UoC}{UoC}{1}{12}{Task1}{
      \label{task:wp1task1}
      Here we will test the WP Task code. 
    }
    \wptask{UoC}{UoC}{6}{9}{Task2}{
      \label{task:wp2task2}
      In this task UZH will integrate the work done in ~\ref{task:wp1test}.
    }    
    \wptask{UoP3}{All other}{9}{12}{Task3}{
      Here all the WP participants will apply the results to...

    }

    
  \end{wpdescription}
  
  % Work Package Deliverable
  \begin{wpdeliverables}
    % Data for the deliverables and milestones  tables
    % syntax: \deliverable[delivery date]{nature}{dissemination
    % level}{description} 
    %
    % nature: R = Report, P = Prototype, D = Demonstrator, O = Other
    % dissemination level: PU = Public, PP = Restricted to other
    % programme participants (including the Commission Services), RE =
    % Restricted to a group specified by the consortium (including the
    % Commission Services), CO = Confidential, only for members of the
    % consortium (including the Commission Services).
    % 
    % \wpdeliverable[date]{R}{PU}{A report on \ldots}

    \wpdeliverable[36]{UoC}{R}{PU}{Report on the definition of the model
      specifications.}\label{dev:wp1specs}
    
    \wpdeliverable[12]{UoP3}{R}{PU}{Report on Feasibility study for the model
      implementation.}\label{dev:wp1implementation}

    \wpdeliverable[24]{UoP2}{R}{PU}{Prototype of model
      implementation.}\label{dev:wp1prototype}

  \end{wpdeliverables}
\end{workpackage}
%%% Local Variables:
%%% mode: latex
%%% TeX-master: "proposal-main"
%%% End:

%%%%%%%%%%%%%%%%%%%%%%%%%%%%%%
%  Work Package Description  %
%%%%%%%%%%%%%%%%%%%%%%%%%%%%%%

\begin{workpackage}{Optimizations}
  \label{wp:opt} %change and use appropriate 
description

  %%%%%%%%%%%%%%%%%% TOP TABLE %%%%%%%%%%%%%%%%%%%%%%%%%%%%%
  % Data for the top table
  \wpstart{1} %Starting Week
  \wpend{36} %End Week
  \wptype{Activity type} %RTD, DEM, MGT, or OTHER

  % Person Weeks per participant (required, max 7, * for leader)  
  % syntax: \personweeks{Participant number}{value}    (not wp leader)
  %     or  \personweeks{Participant short name}{value} (not wp leader)
  %         \personweeks*{Participant number}{value}    (wp leader)
  % for example:
  \personweeks*{UoC}{12}
  % etc.

  \makewptable % Work package summary table
    
  % Work Package Objectives
  \begin{wpobjectives}
    This work package has the following objectives:
    \begin{enumerate}
 \item Optimizing the code in such a way that a trigger thought is recognized 
in realtime.
    \end{enumerate}
  \end{wpobjectives}
  
  % Work Package Description
  \begin{wpdescription}
    % Divide work package into multiple tasks.
    % Use \wptask command
    % syntax: \wptask{leader}{contributors}{start-m}{end-m}{title}{description}   
 
    \wptask{UoC}{UoC}{1}{12}{Task1}{
      \label{task:wp1task1}
      Here we will test the WP Task code. 
    }
    \wptask{UoC}{UoC}{6}{9}{Task2}{
      \label{task:wp2task2}
      In this task UZH will integrate the work done in ~\ref{task:wp1test}.
    }    
    \wptask{UoP3}{All other}{9}{12}{Task3}{
      Here all the WP participants will apply the results to...

    }

    
  \end{wpdescription}
  
  % Work Package Deliverable
  \begin{wpdeliverables}
    % Data for the deliverables and milestones  tables
    % syntax: \deliverable[delivery date]{nature}{dissemination
    % level}{description} 
    %
    % nature: R = Report, P = Prototype, D = Demonstrator, O = Other
    % dissemination level: PU = Public, PP = Restricted to other
    % programme participants (including the Commission Services), RE =
    % Restricted to a group specified by the consortium (including the
    % Commission Services), CO = Confidential, only for members of the
    % consortium (including the Commission Services).
    % 
    % \wpdeliverable[date]{R}{PU}{A report on \ldots}

    \wpdeliverable[36]{UoC}{R}{PU}{Report on the definition of the model
      specifications.}\label{dev:wp1specs}
    
    \wpdeliverable[12]{UoP3}{R}{PU}{Report on Feasibility study for the model
      implementation.}\label{dev:wp1implementation}

    \wpdeliverable[24]{UoP2}{R}{PU}{Prototype of model
      implementation.}\label{dev:wp1prototype}

  \end{wpdeliverables}
\end{workpackage}
%%% Local Variables:
%%% mode: latex
%%% TeX-master: "proposal-main"
%%% End:

%%%%%%%%%%%%%%%%%%%%%%%%%%%%%%
%  Work Package Description  %
%%%%%%%%%%%%%%%%%%%%%%%%%%%%%%

\begin{workpackage}{Real Life Application}
  \label{wp:app} %change and use appropriate description


  %%%%%%%%%%%%%%%%%% TOP TABLE %%%%%%%%%%%%%%%%%%%%%%%%%%%%%
  % Data for the top table
  \wpstart{20} %Starting Week
  \wpend{30} %End Week
  \wptype{Activity type} %RTD, DEM, MGT, or OTHER

  % Person Weeks per participant (required, max 7, * for leader)  
  % syntax: \personweeks{Participant number}{value}    (not wp leader)
  %     or  \personweeks{Participant short name}{value} (not wp leader)
  %         \personweeks*{Participant number}{value}    (wp leader)
  % for example:
  \personweeks*{georgio}{10}
  % etc.

  \makewptable % Work package summary table
    
  % Work Package Objectives
  \begin{wpobjectives}
    This work package has the following objectives:
    \begin{enumerate}
    \item Implement cogniLink to work on a controller of a wheelchair, this task derives from cogniLink as a forked project;
    \item Initiate research about Locked-in syndrome: Find suitiable use-cases and patients.
    \end{enumerate}
  \end{wpobjectives}

  % Work Package Description
  \begin{wpdescription}
    % Divide work package into multiple tasks.
    % Use \wptask command
    % syntax: \wptask{leader}{contributors}{start-m}{end-m}{title}{description}   
 
    \wptask{georgio}{georgio}{20}{22}{Task1}{
      \label{task:wp6task1}
      First Fork of cogniLink.
      A wheelchair controller/helper that can be fed command data as a replacement to the virtual HID will be designed and implemented.
    }
    \wptask{georgio}{georgio}{20}{30}{Task2}{
      \label{task:wp6task2}
      Initiation of formal research about Locked-in syndrome, patients, and practical use-case scenarios.
    }  
  \end{wpdescription}
  
  % Work Package Deliverable
  \begin{wpdeliverables}
    % Data for the deliverables and milestones  tables
    % syntax: \deliverable[delivery date]{nature}{dissemination
    % level}{description} 
    %
    % nature: R = Report, P = Prototype, D = Demonstrator, O = Other
    % dissemination level: PU = Public, PP = Restricted to other
    % programme participants (including the Commission Services), RE =
    % Restricted to a group specified by the consortium (including the
    % Commission Services), CO = Confidential, only for members of the
    % consortium (including the Commission Services).
    % 
    % \wpdeliverable[date]{R}{PU}{A report on \ldots}

    \wpdeliverable[22]{georgio}{R}{PU}{Report about integration with wheelchair.}\label{dev:wp6f1report}
    
    \wpdeliverable[30]{georgio}{P}{PU}{Forking cogniLink and merging changes needed for wheelchair integration.}\label{dev:wp6fork}

    \wpdeliverable[30]{georgio}{R}{PU}{Report about research findings and future steps towards integrating with a locked in patient. }\label{dev:wp6r2research}

    \wpdeliverable[30]{georgio}{R}{PU}{Main Main Report update with full progress accomplished after the end of WP6.}\label{dev:MR0.6}
  
  \end{wpdeliverables}
\end{workpackage}
%%% Local Variables:
%%% mode: latex
%%% TeX-master: "proposal-main"
%%% End:

%%%%%%%%%%%%%%%%%%%%%%%%%%%%%%
%  Work Package Description  %
%%%%%%%%%%%%%%%%%%%%%%%%%%%%%%

\begin{workpackage}{Universal Model, n Commands}
  \label{wp:mUcN} %change and use appropriate 
description

  %%%%%%%%%%%%%%%%%% TOP TABLE %%%%%%%%%%%%%%%%%%%%%%%%%%%%%
  % Data for the top table
  \wpstart{1} %Starting Week
  \wpend{36} %End Week
  \wptype{Activity type} %RTD, DEM, MGT, or OTHER

  % Person Weeks per participant (required, max 7, * for leader)  
  % syntax: \personweeks{Participant number}{value}    (not wp leader)
  %     or  \personweeks{Participant short name}{value} (not wp leader)
  %         \personweeks*{Participant number}{value}    (wp leader)
  % for example:
  \personweeks*{UoC}{12}
  % etc.

  \makewptable % Work package summary table
    
  % Work Package Objectives
  \begin{wpobjectives}
    This work package has the following objectives:
    \begin{enumerate}
    \item To develop a Virtual Human Interface Device;
    \item To develop an  API that gathers raw data from the Cyton board and feeds it 
to a CSV file;
    \item To write code needed to efficiently store and manage datasets;
    \item To write code needed to start training Model 1 on Command A.
    \end{enumerate}
  \end{wpobjectives}
  
  % Work Package Description
  \begin{wpdescription}
    % Divide work package into multiple tasks.
    % Use \wptask command
    % syntax: \wptask{leader}{contributors}{start-m}{end-m}{title}{description}   
 
    \wptask{UoC}{UoC}{1}{12}{Task1}{
      \label{task:wp1task1}
      Here we will test the WP Task code. 
    }
    \wptask{UoC}{UoC}{6}{9}{Task2}{
      \label{task:wp2task2}
      In this task UZH will integrate the work done in ~\ref{task:wp1test}.
    }    
    \wptask{UoP3}{All other}{9}{12}{Task3}{
      Here all the WP participants will apply the results to...

    }

    
  \end{wpdescription}
  
  % Work Package Deliverable
  \begin{wpdeliverables}
    % Data for the deliverables and milestones  tables
    % syntax: \deliverable[delivery date]{nature}{dissemination
    % level}{description} 
    %
    % nature: R = Report, P = Prototype, D = Demonstrator, O = Other
    % dissemination level: PU = Public, PP = Restricted to other
    % programme participants (including the Commission Services), RE =
    % Restricted to a group specified by the consortium (including the
    % Commission Services), CO = Confidential, only for members of the
    % consortium (including the Commission Services).
    % 
    % \wpdeliverable[date]{R}{PU}{A report on \ldots}

    \wpdeliverable[36]{UoC}{R}{PU}{Report on the definition of the model
      specifications.}\label{dev:wp1specs}
    
    \wpdeliverable[12]{UoP3}{R}{PU}{Report on Feasibility study for the model
      implementation.}\label{dev:wp1implementation}

    \wpdeliverable[24]{UoP2}{R}{PU}{Prototype of model
      implementation.}\label{dev:wp1prototype}

  \end{wpdeliverables}
\end{workpackage}
%%% Local Variables:
%%% mode: latex
%%% TeX-master: "proposal-main"
%%% End:


\subsection{List of work packages}
\label{sec:wplist}
\makewplist

\subsection{List of deliverables}\footnote{If your action taking part in the Pilot on Open Research Data, you must include a data management plan as a distinct deliverable within the first 6 weeks of the project.  This deliverable will evolve during the lifetime of the project in order to present the status of the project's reflections on data management. A template for such a plan is available on the Participant Portal (Guide on Data Management).}
\label{sec:deliverables}
\instructions{
\textbf{KEY}\\
\emph{Deliverable numbers in order of delivery dates. Please use the numbering convention $<$WP number$>.<$number of deliverable within that WP$>$.}
\vskip0.2cm
\noindent\emph{For example, deliverable 4.2 would be the second deliverable from work package 4.}
\vskip0.2cm
\noindent\textbf{Type:}\\
\emph{Use one of the following codes:}\\
\indent R: Document, report (excluding the periodic and final reports)\\
\indent DEM: Demonstrator, pilot, prototype, plan designs\\
\indent DEC: Websites, patents filing, press \& media actions, videos, etc.\\
\indent OTHER: Software, technical diagram, etc.
\vskip0.2cm
\noindent\textbf{Dissemination level}:\\
\emph{Use one of the following codes:}\\
\indent PU = Public, fully open, e.g. web\\
\indent CO = Confidential, restricted under conditions set out in Model Grant Agreement\\
\indent CI = Classified, information as referred to in Commission Decision 2001/844/EC.\\
\vskip0.2cm
\noindent\textbf{Delivery date}:\\
Measured in weeks from the project start date (week 1).
}
\makedeliverablelist


\section{Management and risk assessment}
\label{sec:management}
\setcounter{table}{0}
\instructions{
\begin{itemize}
\item Describe the organisational structure and the decision-making ( including a list of
milestones (table 3.2a))
\item Describe any critical risks, relating to project implementation, that the stated project's objectives may not be achieved. Detail any risk mitigation measures. Please provide a table with critical risks identified and mitigating actions (table 3.2b)
\end{itemize}
}



\subsection{List of milestones}
\label{sec:milestones}
\instructions{
\vskip0.2cm
\noindent\textbf{KEY}\\
\textbf{Estimated date}\\
\emph{Measured in weeks from the project start date (week 1)}
\vskip0.2cm
\noindent\textbf{Means of verification}\\
\emph{Show how you will confirm that the milestone has been attained. Refer to indicators if appropriate. For example: a laboratory prototype that is ‘up and running’; software released and validated by a user group; field survey complete and data quality validated.}}

\milestone[24]{Completed simulator development}{Software released and
  validated}{1}
\milestone[36]{Final demonstration}{Application of results}{WP\,\ref{wp:test}}

\makemilestoneslist

\subsection{Critical risks for implementation}
\label{sec:risks}

\criticalrisk{The dedicated chip sent to fabrication is not functional.}{WP\,\ref{wp:test}}{Resort to Software simulations}

\makerisklist

\section{Consortium as a whole} 
\label{sec:consortium}
\instructions{
\emph{The individual members of the consortium are described in a separate section 4. There is no need to repeat that information here.}
\begin{itemize}
\item Describe the consortium. How will it match the project’s objectives? How do the members complement one another (and cover the value chain, where appropriate)? In what way does each of them contribute to the project? How will they be able to work effectively together? 
\item If applicable, describe how the project benefits from any industrial/SME involvement.
\item \textbf{Other countries:} If one or more of the participants requesting EU funding is based in a country that is not automatically eligible for such funding (entities from Member States of the EU, from Associated Countries and from one of the countries in the exhaustive list included in General Annex A of the work programme are automatically eligible for EU funding), explain why the participation of the entity in question is essential to carrying out the project. 
\end{itemize}
}

\section{Resources to be committed} 
\label{sec:resources}
\setcounter{table}{0}
\instructions{
\emph{Please make sure the information in this section matches the costs as stated in the budget table in section 3 of the administrative proposal forms, and the number of person/weeks, shown in the detailed work package descriptions.}
\vskip0.2cm
Please provide the following:\\
\begin{itemize}
\item a table showing number of person/weeks required (table 3.4a)
\item a table showing ``other direct costs'' (table 3.4b) for participants where those costs exceed 15\% of the personnel costs (according to the budget table in section 3 of the administrative proposal forms)  
\end{itemize}
}

\subsection{Summary of staff efforts}
\instructions{Table 3.4a: \emph{Please indicate the number of person/weeks over the whole duration of the planned work, for each work package, for each participant. Identify the work-package leader for each WP by showing the relevant person-week figure in bold.}}
\makesummaryofefforttable

\subsection{‘Other direct cost’ items (travel, equipment, other goods and services, large research infrastructure)}
\instructions{
Please provide a table of summary of costs for each participant, if the sum of the costs for ``travel'', ``equipment'', and ``goods and services'' exceeds 15\% of the personnel costs for that participant (according to the budget table in section 3 of the proposal administrative forms).
}

\costsTravel{UoC}{2500}{3 pairwise meetings for 2 people, 2 conferences for 3 people, 3 internal project meetings for 3 people}
\costsEquipment{UoC}{3000}{CAD workstation for chip design}
\costsOther{UoC}{60000}{Fabrication of 2 VLSI chips}
\costsOther{UoP2}{40000}{Fabrication of prototype PCBs}

\makecoststable


\instructions{
Please complete the table below for all participants that would like to declare costs of large research infrastructure under Article 6.2 of the General Model Agreement, irrespective of the percentage of personnel costs. Please indicate (in the justification) if the beneficiary's methodology for declaring the costs for large research infrastructure has already been positively assessed by the Commission.\\
Note: Large research infrastructure means research infrastructure of a total value of at least EUR 20 million, for a beneficiary. More information and further guidance on the direct costing for the large research infrastructure is available in the H2020 Online Manual on the Participant Portal.
}

\costslri{UoP3}{400000}{Synchrotron}
\costslri{UoC}{400000}{Synchrotron}

\makelritable
 % Section I

\bibliographystyle{apacite}       %References go to end of Section I
\bibliography{refs}

\clearpage

\chapter{Ethics and Security}
\label{cha:ethics}
\instructions{
\textit{This section is not covered by the page limit.}
}

\section{Ethics}
\label{sec:ethics}
\instructions{
If you have entered any ethics issues in the ethical issue table in the administrative proposal forms, you must:
\begin{itemize}
\item submit an ethics self-assessment, which:
\begin{itemize}
\item describes how the proposal meets the national legal and ethical requirements of the country or countries where the tasks raising ethical issues are to be carried out; 
\item explains in detail how you intend to address the issues in the ethical issues table, in particular as regard:
\begin{itemize}
\item research objectives (e.g. study of vulnerable populations, dual use, etc.)
\item research methodology (e.g. clinical trials, involvement of children and related consent procedures, protection of any data collected, etc.) 
\item the potential impact of the research (e.g. dual use issues, environmental damage, stigmatisation of particular social groups, political or financial retaliation, benefit-sharing,  malevolent use, etc.).
\end{itemize}
\end{itemize}
\item provide the documents that you need under national law(if you already have them), e.g.:
\begin{itemize}
\item an ethics committee opinion;
\item the document notifying activities raising ethical issues or authorising such activities;
\end{itemize}
\end{itemize}
\textit{\indent If these documents are not in English, you must also submit an English summary of them (containing, if available, the conclusions of the committee or authority concerned).}
\vskip0.2cm
\textit{If you plan to request these documents specifically for the project you are proposing, your request must contain an explicit reference to the project title.}
}

\section{Security}\footnote{Article 37.1 of the Model Grant Agreement: Before disclosing results of activities raising security issues to a third party (including affiliated entities), a beneficiary must inform the coordinator -- which must request written approval from the Commission/Agency. Article 37.2: Activities related to ``classified deliverables'' must comply with the ``security requirements'' until they are declassified. Action tasks related to classified deliverables may not be subcontracted without prior explicit written approval from the Commission/Agency. The beneficiaries must inform the coordinator -- which must immediately inform the Commission/Agency -- of
any changes in the security context and --if necessary -- request for Annex 1 to be amended (see Article 55).
}
\label{sec:security}
\instructions{
Please indicate if your project will involve:
\begin{itemize}
\item activities or results raising security issues: (YES/NO)
\item ``EU-classified information'' as background or results: (YES/NO)
\end{itemize}
}
 % Section II

\appendix

%% Proposal appendix

 % Appendix

\backmatter

\end{document}
